Обычно теоретическая или практическая часть,
в~которой подробно излагаются собственные результаты автора.
Теоремы приводятся с доказательствами.

\paragraph{Формулы}
внутри текста, даже очень короткие, окружаются знаками доллара~\verb'$':

\begin{center}
    \begin{tabular}{l|l}
        \verb'число $-3.14$' & число $-3.14$ --- верно\\
        \verb'число -3.14'   & число -3.14 --- неверно   \\
        \verb'объект~$x$'    & объект~$x$ --- верно    \\
        \verb'объект x'      & объект x --- неверно
    \end{tabular}
\end{center}

Выключные формулы без номера окружаются скобками \verb'\[' и \verb'\]'.
\[
    y(x,\alpha) =
    \begin{cases}
        -1, & \text{если } f(x,\alpha)<0;  \\
        +1, & \text{если } f(x,\alpha)\geq 0.
    \end{cases}
\]

Выключные формулы с~номером окружаются командами
\verb'\begin{equation}' и~\verb'\end{equation}'.
Команда \verb'\label{'\emph{name}\verb'}' между ними
задаёт метку формулы.
Русские буквы в~именах меток~\emph{name} не~допустимы.
Метка позволяет ссылаться на~формулу командой
\verb'\eqref{'\emph{name}\verb'}',
например команда \verb'\eqref{eq:cases}' даёт~\eqref{eq:cases}.
\begin{equation}
\label{eq:cases}
    y(x,\alpha) =
    \begin{cases}
        -1, & \text{если } f(x,\alpha)<0;  \\
        +1, & \text{если } f(x,\alpha)\geq 0.
    \end{cases}
\end{equation}

Образец выключной формулы, разбитой на две строки окружением \verb'align':
\begin{align}
    R'_N(F)
        = \frac1N \sum_{i=1}^N
        \Bigl(
            & P(+1\cond x_i) C\bigl(+1,F(x_i)\bigr)+{}
        \notag % подавили номер у первой строки
    \\
        \label{eq:R(F)}
            & P(-1\cond x_i) C\bigl(-1,F(x_i)\bigr)
        \Bigr).
\end{align}

\paragraph{Таблицы}
создаются окружением \verb'tabular'
и~оформляются как плавающие с~помощью окружения  \verb'table'.
Положением плавающих таблиц на~странице
управляет необязательный аргумент команды
\verb'\begin{table}'.
Подпись делается \emph{над таблицей} командой \verb'\caption', см.~таблицу~\ref{TabExample}.
Команда \verb'\label', определяющая ссылку на~номер таблицы,
обязана идти после \verb'\caption'.
Если таблица не умещается по~ширине страницы,
то~можно уменьшить шрифт на \verb'\small' или даже \verb'\footnotesize',
либо уменьшить интервалы между колонками:
\verb'\tabcolsep=2pt'.

\begin{table}[t]%\small
    \caption{Подпись размещается над таблицей.}
    \label{TabExample}
    \centering\medskip%\tabcolsep=2pt%\small
    \begin{tabular}{lrrr}
    \headline
        Задача
            & \multicolumn{1}{c}{CCEL}
            & \multicolumn{1}{c}{boosting} \\
    \headline
        {\tt Cancer}
            & $\mathbf{3.46}  \pm 0.37$ (3.16)
            & $4.14 \pm 1.48$ \\
        {\tt German}
            & $\mathbf{25.78} \pm 0.65$ (1.74)
            & $29.48 \pm 0.93$ \\
        {\tt Hepatitis}
            & $18.38 \pm 1.43$ (2.87)
            & $19.90 \pm 1.80$ \\
    \hline
    \end{tabular}
\end{table}

%\paragraph{Списки}
%оформляются стандартными окружениями \verb|enumerate| или \verb|itemize|.
%В~стиле \verb|mtaithesis.sty| определено окружение \verb|enumerate*|
%для списков, в~которых, согласно правилам русской пунктуации:
%\begin{enumerate*}
%\item
%    номера отделяются скобкой;
%\item
%    пункты начинаются со~строчной буквы;
%\item
%    и~заканчиваются точкой с~запятой.
%\end{enumerate*}
%
%Этот список удобен для перечисления коротких пунктов, умещающихся в~одну строку.
%Если пункты более длинные, то~лучше воспользоваться стандартным окружением \verb|enumerate|,
%указав после \verb|\begin{enumerate}|
%команду \verb|\afterlabel)|,
%которая переопределит точку после номера на~скобку.

\paragraph{Графические иллюстрации}
могут быть подготовлены в~любом графическом формате,
в~частности, \verb'BMP', \verb'PNG' или \verb'EPS'.

Желательно, чтобы рисунки были чёрно"=белыми или grayscale (оттенки серого).
При чёрно"=белой печати передача цвета плохо предсказуема.
Жёлтый цвет, как правило, не виден.
Синий, зелёный и~красный могут оказаться неотличимыми.

Рисунки вставляются с~помощью окружения \verb'figure'
и~могут разрывать текст в~любом месте.
Положением плавающих рисунков на~странице
управляет необязательный аргумент команды
\verb'\begin{figure}'.
Подпись делается \emph{под рисунком} командой \verb'\caption',
см.\,рис.\,\ref{TabExample}.

\begin{figure}[t]
    \centering
    \includegraphics[width=110mm]{figExample.eps}
    \caption{Подпись должна размещаться под рисунком.
        ВНИМАНИЕ! Красные и~синие линии при чёрно"=белой печати будут выглядеть как чёрные.}
    \label{FigExample}
\end{figure}

Оси на графике должны быть подписаны.
Если на графике несколько кривых, обязательно должна быть легенда.
В~подписи под графиком обычно пишут, что за зависимость изображена,
и~при каких условиях эксперимента был получен данный график.

Для фотографических изображений лучше подходит формат \verb'JPEG',
для растровых "--- \verb'PNG'.
Рекомендуется тот и~другой сначала перевести в~\verb'EPS' (Encapsulated PostScript).
Это делается с~помощью утилиты \verb'bmeps', входящей в пакет MiK\TeX:
\begin{verbatim}
    bmeps -c -t jpg myfig.jpeg myfig.eps
    bmeps -c -t png myfig.png myfig.eps
\end{verbatim}

Для векторной графики лучше найти способ записать изображение непосредственно в~формате~\verb'EPS'.
Например, в MATLAB имеется функция сохранения графика~в~\verb'EPS'.
Если~же делать скриншот экрана и~записывать изображение через растровый формат,
качество изображения будет скверным.
Преобразование файла растрового изображения в~\verb'EPS' не делает его <<настоящим>> векторным,
просто каждый пиксел рисуется прямоугольничком.

\paragraph{Алгоритмы}
оформляются в~стиле псевдокода с~помощью окружения \verb'algorithmic',
внутри которого определены стандартные ключевые слова
\verb'\IF', \verb'\FOR', \verb'\WHILE', и~др.,
которые при печати дают, соответственно,
\algKeyword{если}, \algKeyword{для}, \algKeyword{пока}, и~т.\,д.
Шаги алгоритма нумеруются автоматически,
и~на~них можно ссылаться,
см.~шаг~\ref{algCalcU} алгоритма~\ref{AlgExample}.

\begin{algorithm}[t]
    \begin{algorithmic}[1]
    \caption{Показаны все допустимые управляющие конструкции.}
    \label{AlgExample}
        \REQUIRE $x, y$;
        \ENSURE $z = F(x,y)$;
        \STATE инициализация: $b := a$;
        \FOR{$i=1,\dots,n$}
            \FORALL{$w \in W$ таких, что $w \geq 0$}
                \REPEAT
                    \STATE важный шаг: вычисление вектора~$u_i$;
                    \label{algCalcU}
                \UNTIL{$\|u_i-u_{i-1}\|>\epsilon$};
            \ENDFOR
        \ENDFOR
        \IF{$a>0$}
            \WHILE{$W\neq\varnothing$}
                \STATE $W := W-\{a\}$;
            \ENDWHILE
        \ELSIF{$a=0$}
            \LOOP[бесконечный цикл]
                \STATE при определённых условиях \EXIT;
            \ENDLOOP
        \ELSE[при $a<0$]
            \STATE $a:=1$;
        \ENDIF
    \end{algorithmic}
\end{algorithm}

\paragraph{Рисование графов}
с~помощью окружения \verb'network' из~пакета~\Xy-pic.
В~стиле \verb'mtaithesis.sty' определены две вспомогательные команды.
Команда \verb'\nnNode' задаёт имя и~координаты вершины,
команда \verb'\nnLink' связывает две ранее поименованных вершины.
Внешний вид вершин и~связей задаётся средствами пакета~\Xy-pic:
\[
    \begin{network}
        \nnNode[x1](0,7)    {+[o][F]{x^1}}
        \nnNode[x2](0,2)    {+[o][F]{x^2}}
        \nnNode[dd](0,-3)   {{\cdots}}
        \nnNode[xn](0,-7)   {+[o][F]{x^n}}
        \nnNode[1](7,-9 )   {+[o][F]{1}}
        \nnNode[sum](14,0)  {++[F-]{\sum}}
        \nnNode[sig](21,0)  {++[F-]{\sigma}}
        \nnNode[y](30,0)    {+[o][F]{y}}
        \nnLink[x1,sum]     {@{->}|{w_1}}
        \nnLink[x2,sum]     {@{->}|{w_2}}
        \nnLink[dd,sum]     {@{}|{\dots}}
        \nnLink[xn,sum]     {@{->}|{w_n}}
        \nnLink[1,sum]      {@{->}@/_3ex/|{w_0}}
        \nnLink[sum,sig]    {@{->}}
        \nnLink[sig,y]      {@{->}}
    \end{network}
\]

\subparagraph{Некоторые правила типографики.}
Скобки всех видов набираются вплотную к тексту, который они окружают.
Знаки препинания набираются
слитно с~предшествующим текстом и~отдельно от~последующего.

Кавычки делаются знаками меньше--больше: \verb'<<'\emph{текст}\verb'>>'.
Использовать в~роли кавычек символ \verb'"' нельзя!

Многоточия в~тексте и~формулах делаются командой \verb'\dots'.

Тире делается командой \verb'"---'
и~отделяется от~предшествующего и~последующего текста пробелами:
\verb*'Знание "--- сила'.

В~длинных словах с~дефисом, таких, как <<счётно"=аддитивно>>,
дефис делается командой \verb'"=', иначе слово не~будет переноситься:
\verb'счётно"=аддитивно'.
Команда"=дефис \verb'"~' запрещает переносы:
$F$"~пре\-образование,
\verb'$F$"~пре\-образование'.

Неразрывный пробел~\verb'~' ставится
между коротким предлогом и~последующим словом,
а~также между очень короткой формулой и~связанным с~ней по~смыслу словом:
\verb'число~$N$ в~$k$~раз' \verb'больше, чем~$n$'.

Между идущими подряд формулами лучше вставлять дополнительный пробел:
\begin{center}
\begin{tabular}{l|l}
    \verb'$a=1,b=2$'       & $a=1,b=2$ ~~~--- плохо \\
    \verb'$a=1$, $b=2$'    & $a=1$, $b=2$ ~~~--- получше \\
    \verb'$a=1$,\: $b=2$'  & $a=1$,\: $b=2$  ~~~--- хорошо \\
    \verb'$a=1$,\; $b=2$'  & $a=1$,\; $b=2$  ~~~--- хорошо
\end{tabular}
\end{center}

В~русскоязычной математической литературе принято
при переносе формулы на новую строку дублировать на новой строке знак математического оператора.
В~стилевом файле \verb|mtaithesis.sty| для этого определена команда \verb|\brop|.
Пример:
$\frac12 \sqrt{b^2-4ac} \brop= \sqrt{\bigl(\tfrac b2\bigr)^2 - ac}$.

Иногда в~формуле надо убрать пробелы вокруг знака операции.
Например, если знак $\times$ используется не~как произведение,
а~для указания размеров матрицы или растрового изображения,
то~он не~должен окружаться пробелами:
\begin{center}
\begin{tabular}{l|l}
    \verb'$640\times 480$'  & $640\times 480$ ~~~--- плохо \\
    \verb'$640{\times}480$' & $640{\times}480$ ~~~--- хорошо
\end{tabular}
\end{center}

Дополнительный пробел \verb'\quad' рекомендуется вставлять
между выражениями, идущими через запятую в~выключной формуле.

Короткий пробел \verb'\,' ставится в~инициалах и~сокращениях:
\begin{center}
\begin{tabular}{l|l}
    \verb'т. е.',  \verb'и т. д.'    & т. е.,  и т. д. ~~~--- плохо \\
    \verb'т.\,е.',  \verb'и~т.\,д.'  & т.\,е.,  и~т.\,д. ~~~--- хорошо\\
    \verb'Г.С.Осипов'              & Г.С.Осипов ~~~--- плохо \\
    \verb'Г.\,С.\,Осипов'          & Г.\,С.\,Осипов ~~~--- хорошо
\end{tabular}
\end{center}

Не~желательно использовать жирный шрифт для выделения
\emph{важных слов} или \emph{терминов}.
Это делается командой \verb'\emph{'\emph{текст}\verb'}'.

\paragraph{Разумное форматирование} исходного кода
заметно облегчает работу с~текстом для Вас и~научного руководителя.
По~возможности придерживайтесь нескольких простых правил:
\begin{itemize}
\item
    начинайте каждое предложение с~новой строки;
\item
    команды \verb'\begin', \verb'\end', \verb'$$', \verb'\[', \verb'\]',
    \verb'\section', \verb'\subsection', \verb'\paragraph'
    \verb'\item', \verb'\bibitem', \verb'\par', \verb'\label'
    набирайте отдельной строкой;
\item
    внутритекстовые формулы, за~исключением совсем коротких,
    набирайте отдельной строкой;
\item
    описания длинных формул разбивайте на строки;
    используйте форматирование исходного текста с~отступами,
    набирая отдельной строкой команды скобок
    \verb'\left', \verb'\right', и~т.\,п.,
    как показано в~Примере~\ref{exExample}.
    %Это облегчает понимание и~правку формул в~\TeX-файлах.
\end{itemize}

\begin{Example}
\label{exExample}
    Без <<правильного>> форматирования
    было~бы легко запутаться в~скобках и~похожих частях формулы:
    \begin{align*}
    R'_N(F)
        = \frac1N \sum_{i=1}^N
        \Bigl(&
            P(+1\cond x_i) C\bigl(+1,F(x_i)\bigr)+{}
        \\&
            P(-1\cond x_i) C\bigl(-1,F(x_i)\bigr)
        \Bigr).
    \end{align*}
\end{Example}
\begin{verbatim}
    \begin{align*}
       R'_N(F)
       = \frac1N \sum_{i=1}^N
       \Bigl( &
          P(+1\cond x_i)C\bigl(+1,F(x_i)\bigr)+{}
       \\ &
          P(-1\cond x_i)C\bigl(-1,F(x_i)\bigr)
       \Bigr).
    \end{align*}
\end{verbatim}