% В~заключении конспективно перечисляются результаты, выносимые на~защиту.

% Пример:

% \paragraph{Основные результаты работы}
% \begin{itemize}
% \item
%     Предложен новый подход к\dots Доказано, что\dots 
% \item
%     Разработаны и реализованы алгоритмы\dots
% \item
%     Проведены вычислительные эксперименты\dots,
%     которые подтвердили / опровергли гипотезу о~том, что\dots
% \end{itemize}

% В~заключении можно также перечислить предполагаемые направления дальнейших исследований.

В настоящей работе были предложены и описаны объектно-центричные модели мира в обучении с подкреплением для алгоритмов модельного подхода, решающих задачу визуального роботизированного контроля. За счет предположения о работе только с манипулятором и объектом, стало возможным внести в структуру в явном виде предположения о причинно-следственных связях в среде. Благодаря подобному моделированию, а также благодаря модификациям, свойственным только объектно-центричным моделям мира, было достигнуто улучшение обобщающей способности базового алгоритма.

Предложенное направление исследований представляется весьма перспективным, поскольку применение объектно-центричного подхода в других областях науки уже доказало его состоятельность. В настоящей работе была показана возможность добиться улучшений обобщающей способности алгоритмов, но при этом присутствовал ряд существенных ограничений, ограничивающих применение алгоритма для реальных задач.

Таким образом, для дальнейших исследований можно выделить следующие направления:
\begin{enumerate}
    \item избавиться от необходимости использовать истинные сегментационные маски в алгоритме;
    \item исследовать возможности интеллектуальной активации вектора влияния в моменты взаимодействия манипулятора с объектом;
    \item рассмотреть случай с большим количеством объектов и/или манипуляторов;
    \item рассмотреть возможные применения объектно-центричного подхода для улучшения иных качеств алгоритмов обучения с подкреплением.
\end{enumerate}