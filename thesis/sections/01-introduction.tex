% Введение объясняет мотивацию работы:
% где возникает данная задача,
% почему её решение так важно,
% как её решали до сих пор,
% в~чём недостатки этих решений,
% и~что нового предлагает автор.
% Введение лучше писать напоследок, так как
% в~ходе работы обычно происходит переосмысление постановки задачи.

% Вводятся на~неформальном уровне основные понятия,
% необходимые для понимания постановки задачи.
% Определяются цели исследования и~формулируется постановка задачи.

% Во~введении можно привести краткий анализ источников информации. 
% Однако если литературный обзор большой, ему лучше посвятить отдельный раздел.

% В~конце введения даётся краткое содержание работы по~разделам,
% при этом отмечается, какие подходы, методы, алгоритмы
% предлагаются автором впервые.
% Перечисляются основные результаты и~1--2 самых важных вывода работы.

% Введение может быть близко по содержанию к~тексту доклада на защите.


Обучение с подкреплением является одним из многообещающих направлений изучения искусственного интеллекта. 
Данная область науки занимается задачами последовательного принятия решений. Достаточно большое количество задач в реальной жизни подходит под это описание, например, контроль действий манипулятора, управление беспилотным автомобилем или компьютерные игры. 
В каждой из подобных задач можно выделить две основные сущности, взаимодействующие друг с другом - агента и среду. 
Агент может получать от среды информацию различного типа на каждом шаге - например, вектор позиций конечностей манипулятора или картинку с камеры автомобиля. 
Как и в остальных областях машинного обучения, работа с визуальными данными требует особых решений. 
В последние годы были достигнуты значительные успехи в решении подобного рода задач - в частности, в 2015 году были достигнуты  результаты, сравнимые с человеческими в играх Atari, а позже -  побиты результаты игры в Starcraft 2 и Dota 2. 
Несмотря на подобные успехи, применение алгоритмов обучения с подкреплением в реальной жизни достаточно ограничено по причине имеющихся недостатков большинства подходов. 
Для алгоритмов RL свойственно требовать очень большое количество взаимодействия со средой, они обладают плохой обобщающей способностью а также плохо интерпретируются. 
Обобщающая способность является чрезвычайно важным свойством моделей, от которого напрямую зависит возможность применения алгоритма для реальных задач, а не только в искусственных средах.

Одним из способов преодолеть данные недостатки является построение модели среды, с которой взаимодействует агент. 
Улучшение качества модели среды позволяет алгоритму проще адаптироваться к новым задачам в той же среде. 
Однако без специальных модификаций, направленных на улучшение обобщающих способностей, данные алгоритмы все равно показывают слабые обобщающие способности.

За последние годы было предложено много методов улучшения обобщающих способностей алгоритмов в этой области, доминирующим направлением в них является предобучение алгоритма с целью максимизации определенного внутреннего информационного показателя качества. 
Максимизируя их, алгоритм более эффективно исследует среду и выучивает более богатую и устойчивую модель среды, что позволяет ему избежать переобучения для определенной функции награды. 
Данный подход - предтренировка алгоритма - был успешно адаптирован из другой области машинного обучения, машинного зрения.

Во многих задачах, рассматриваемых в машинном зрении, таких как генерация сцен и генерация видео, в последние годы успешно применяется объектно-центричный подход, заключающийся в выделении на изображении отдельных сущностей. 
Введение подобного рода структуры позволяет упростить моделирование отдельных компонент изображения и сделать этот процесс более интерпретабельным. 
Также данный подход выглядит весьма уместным и перспективным в случаях, когда объекты зависят друг от друга, позволяя моделировать причинно-следственные связи.

В обучении с подкреплением объектно-центричный подход известен достаточно давно, однако исследований в данном направлении велось достаточно мало. Несмотря на то, что не во всех задачах можно с успехом выделить отдельные объектные сущности, в задачах роботизированного контроля переход к подобного рода абстракциям достаточно тривиален. 
В среде подразумевается присутствие манипулятора и, опционально, некоторого количества объектов, с которыми манипулятор может взаимодействовать.

Модели мира могут получать большие преимущества от переиспользования смоделированной или заранее заданной структуры среды \cite{WM}.
В связи с этим представляется достаточно перспективным внедрение объектно-центричного подхода в алгоритмах, основанных на модели среды, решающих задачи роботизированного контроля, в которых агент получает в качестве наблюдений изображения среды.
Тогда как его влияние на скорость обучения в решении конкретных задач не очевидно, обобщающие способности алгоритмов должны возрасти за счет отдельного моделирования динамик объектов на изображении.
% - задача, в которой, управляя манипулятором и получая в качестве наблюдений изображения среды, агенту необходимо достичь заранее поставленную цель, например переместить кубик в определенное место. 