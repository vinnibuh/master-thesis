В данной работе рассматриваются задача улучшения обобщающей способности алгоритмов модельного подхода в обучении с подкреплением, решающих задачу визуального роботизированного контроля. Предложены различные объектно-центричные модели мира агентов, учитывающие причинно-следственные связи между манипулятором и объектом в среде. Рассматривается семейство моделей, выделяющее в явном виде влияние манипулятора на объект и проанализированы их качества и потенциал для разных сред. Для оценки качества предложенных алгоритмов проведены эксперименты в двух наборах сред, CasualWorld и MetaWorld. Полученные результаты позволяют утверждать, что рассмотренные модели мира позволяют добиться большей обобщающей способности, чем базовый алгоритм модельного подхода.