\begin{frame}
\frametitle{Модель мира CEMA}
% TODO: titles are hanging out of images
% TODO: maybe draw commentaries
% TODO: draw an arrow

\begin{columns}[b]
\begin{column}{0.48\linewidth}
\only<1-2>{
    Модель мира Dreamer\cite{Dreamer}:
    \vspace{0.45cm}
    
    \footnotesize{
    \begin{itemize}
    \setlength\itemsep{1ex}
    \setlength\topsep{1.5ex}
        \item $s_t \sim p(s_t \mid s_{t-1}, a_{t-1})$ модель динамики среды
        \item $s_t \sim q(s_t \mid s_{t-1}, a_{t-1}, o_t)$ модель вывода
        \item $p(o_t \mid s_t)$ модель изображения
        \item $p(r_t \mid s_t)$ модель награды
    \end{itemize} }
    \vspace{0.4cm}
    
    \begin{figure}
        \resizebox{.9\linewidth}{!}{\begin{tikzpicture}[scale=2.1]
    \begin{scope}[yshift=8cm,node distance=5em and 5em]
	\path [draw, fill={rgb:red,1;white,3}, rounded corners] (-2.2,-1.5) rectangle (2.0,1.5) {};
	\node[] (rob_dyn) at (-1.0,-1.3) {\LARGE Динамика среды};
    \node[mystyle, fill={rgb:white,1},            ] (st) {\LARGE $s_t$};
    \node[mystyle, fill={rgb:white,1}, left=of st] (stm1) {\LARGE $s_{t-1}$};
    \node[right=of st] (stp1) {};
    \node[mystyle, fill={rgb:white,1}, above left=4em and 3em of st] (atm1) {\LARGE $a_{t-1}$};
    \node[mystyle, fill={rgb:black,1;white,3}, above right=4em and 3em of st] (ot) {\LARGE $o_t$};
    \node[mystyle, fill={rgb:black,1;white,3}, below right=4em and 3em of st] (rt) {\LARGE $r_t$};
    
    \draw[myline] (stm1) -- (st);
	\draw[myline] (atm1) -- (st);
	\draw[myline] (st) -- (ot);
	\draw[myline] (st) -- (rt);
	\draw[myline] (st) -- (stp1);
	
	\draw[myline, dashed, color={rgb:red,2;black,1}] (stm1) edge[bend left] (st);
	\draw[myline, dashed, color={rgb:red,2;black,1}] (ot) edge[bend left] (st);
	\end{scope}
\end{tikzpicture}}
    \end{figure}
}
\only<3->{
    Части модели мира:
    
    \vspace{0.5cm}
    \footnotesize{\textbf{Генеративная} модель прямой причинности:}
    \scriptsize{
    \begin{itemize}
    \setlength\itemsep{1ex}
    \setlength\topsep{1.5ex}
        \item<3-> $s_t \sim p(s_t \mid s_{t-1}, a_{t-1})$ модель динамики скрытых состояний
        \item<4-> $u_t \sim p(u_t \mid s_t, c)$ модель прямой причинности
        \item<5-> $s'_t \sim p(s'_t\mid s'_{t-1}, u_t)$ модель обновления 
    \end{itemize} }
    \vspace{0.4cm}
    \footnotesize{Обратная причинная модель (\textbf{модель вывода})}:
    \scriptsize{
    \begin{itemize}
    \setlength\itemsep{1ex}
    \setlength\topsep{1.5ex}
        \item<3-> $s_t \sim q(s_t \mid s_{t-1}, a_{t-1}, o_t)$ модель вывода состояния робота 
        \item<8-> $s'_t \sim q(s'_t \mid o'_t)$ модель вывода объектного состояния
        \item<9-> $u_t \sim q(u_t \mid s_t, s'_t)$ модель обратной причинности
    \end{itemize} }
    \vspace{0.4cm}
    \footnotesize{Выходы нейронной сети}:
    \scriptsize{
    \begin{itemize}
    \setlength\itemsep{1ex}
    \setlength\topsep{1.5ex}
        \item<3-> $p(o_t \mid s_t)$ субъектный декодер 
        \item<5-> $p(o'_t \mid s'_t)$ объектный декодер
        \item<6-> $p(r_t \mid u_t)$ декодер награды
        \item<10->$\pi(a_t \mid s_t, c)$, $v(s_{t}, u_{t}, s'_{t})$ стратегия и функция полезности
    \end{itemize} }
    \vspace{0.1cm}
}
\end{column}

\begin{column}{0.48\linewidth}
\only<2-10>{
    Модель мира CEMA:
    \begin{figure}
        \resizebox{.9\linewidth}{!}{\begin{frame}
\frametitle{Модель мира CEMA}
% TODO: titles are hanging out of images
% TODO: maybe draw commentaries
% TODO: draw an arrow

\begin{columns}[b]
\begin{column}{0.48\linewidth}
\only<1-2>{
    Модель мира Dreamer\cite{Dreamer}:
    \vspace{0.45cm}
    
    \footnotesize{
    \begin{itemize}
    \setlength\itemsep{1ex}
    \setlength\topsep{1.5ex}
        \item $s_t \sim p(s_t \mid s_{t-1}, a_{t-1})$ модель динамики среды
        \item $s_t \sim q(s_t \mid s_{t-1}, a_{t-1}, o_t)$ модель вывода
        \item $p(o_t \mid s_t)$ модель изображения
        \item $p(r_t \mid s_t)$ модель награды
    \end{itemize} }
    \vspace{0.4cm}
    
    \begin{figure}
        \resizebox{.9\linewidth}{!}{\begin{tikzpicture}[scale=2.1]
    \begin{scope}[yshift=8cm,node distance=5em and 5em]
	\path [draw, fill={rgb:red,1;white,3}, rounded corners] (-2.2,-1.5) rectangle (2.0,1.5) {};
	\node[] (rob_dyn) at (-1.0,-1.3) {\LARGE Динамика среды};
    \node[mystyle, fill={rgb:white,1},            ] (st) {\LARGE $s_t$};
    \node[mystyle, fill={rgb:white,1}, left=of st] (stm1) {\LARGE $s_{t-1}$};
    \node[right=of st] (stp1) {};
    \node[mystyle, fill={rgb:white,1}, above left=4em and 3em of st] (atm1) {\LARGE $a_{t-1}$};
    \node[mystyle, fill={rgb:black,1;white,3}, above right=4em and 3em of st] (ot) {\LARGE $o_t$};
    \node[mystyle, fill={rgb:black,1;white,3}, below right=4em and 3em of st] (rt) {\LARGE $r_t$};
    
    \draw[myline] (stm1) -- (st);
	\draw[myline] (atm1) -- (st);
	\draw[myline] (st) -- (ot);
	\draw[myline] (st) -- (rt);
	\draw[myline] (st) -- (stp1);
	
	\draw[myline, dashed, color={rgb:red,2;black,1}] (stm1) edge[bend left] (st);
	\draw[myline, dashed, color={rgb:red,2;black,1}] (ot) edge[bend left] (st);
	\end{scope}
\end{tikzpicture}}
    \end{figure}
}
\only<3->{
    Части модели мира:
    
    \vspace{0.5cm}
    \footnotesize{\textbf{Генеративная} модель прямой причинности:}
    \scriptsize{
    \begin{itemize}
    \setlength\itemsep{1ex}
    \setlength\topsep{1.5ex}
        \item<3-> $s_t \sim p(s_t \mid s_{t-1}, a_{t-1})$ модель динамики скрытых состояний
        \item<4-> $u_t \sim p(u_t \mid s_t, c)$ модель прямой причинности
        \item<5-> $s'_t \sim p(s'_t\mid s'_{t-1}, u_t)$ модель обновления 
    \end{itemize} }
    \vspace{0.4cm}
    \footnotesize{Обратная причинная модель (\textbf{модель вывода})}:
    \scriptsize{
    \begin{itemize}
    \setlength\itemsep{1ex}
    \setlength\topsep{1.5ex}
        \item<3-> $s_t \sim q(s_t \mid s_{t-1}, a_{t-1}, o_t)$ модель вывода состояния робота 
        \item<8-> $s'_t \sim q(s'_t \mid o'_t)$ модель вывода объектного состояния
        \item<9-> $u_t \sim q(u_t \mid s_t, s'_t)$ модель обратной причинности
    \end{itemize} }
    \vspace{0.4cm}
    \footnotesize{Выходы нейронной сети}:
    \scriptsize{
    \begin{itemize}
    \setlength\itemsep{1ex}
    \setlength\topsep{1.5ex}
        \item<3-> $p(o_t \mid s_t)$ субъектный декодер 
        \item<5-> $p(o'_t \mid s'_t)$ объектный декодер
        \item<6-> $p(r_t \mid u_t)$ декодер награды
        \item<10->$\pi(a_t \mid s_t, c)$, $v(s_{t}, u_{t}, s'_{t})$ стратегия и функция полезности
    \end{itemize} }
    \vspace{0.1cm}
}
\end{column}

\begin{column}{0.48\linewidth}
\only<2-10>{
    Модель мира CEMA:
    \begin{figure}
        \resizebox{.9\linewidth}{!}{\begin{frame}
\frametitle{Модель мира CEMA}
% TODO: titles are hanging out of images
% TODO: maybe draw commentaries
% TODO: draw an arrow

\begin{columns}[b]
\begin{column}{0.48\linewidth}
\only<1-2>{
    Модель мира Dreamer\cite{Dreamer}:
    \vspace{0.45cm}
    
    \footnotesize{
    \begin{itemize}
    \setlength\itemsep{1ex}
    \setlength\topsep{1.5ex}
        \item $s_t \sim p(s_t \mid s_{t-1}, a_{t-1})$ модель динамики среды
        \item $s_t \sim q(s_t \mid s_{t-1}, a_{t-1}, o_t)$ модель вывода
        \item $p(o_t \mid s_t)$ модель изображения
        \item $p(r_t \mid s_t)$ модель награды
    \end{itemize} }
    \vspace{0.4cm}
    
    \begin{figure}
        \resizebox{.9\linewidth}{!}{\begin{tikzpicture}[scale=2.1]
    \begin{scope}[yshift=8cm,node distance=5em and 5em]
	\path [draw, fill={rgb:red,1;white,3}, rounded corners] (-2.2,-1.5) rectangle (2.0,1.5) {};
	\node[] (rob_dyn) at (-1.0,-1.3) {\LARGE Динамика среды};
    \node[mystyle, fill={rgb:white,1},            ] (st) {\LARGE $s_t$};
    \node[mystyle, fill={rgb:white,1}, left=of st] (stm1) {\LARGE $s_{t-1}$};
    \node[right=of st] (stp1) {};
    \node[mystyle, fill={rgb:white,1}, above left=4em and 3em of st] (atm1) {\LARGE $a_{t-1}$};
    \node[mystyle, fill={rgb:black,1;white,3}, above right=4em and 3em of st] (ot) {\LARGE $o_t$};
    \node[mystyle, fill={rgb:black,1;white,3}, below right=4em and 3em of st] (rt) {\LARGE $r_t$};
    
    \draw[myline] (stm1) -- (st);
	\draw[myline] (atm1) -- (st);
	\draw[myline] (st) -- (ot);
	\draw[myline] (st) -- (rt);
	\draw[myline] (st) -- (stp1);
	
	\draw[myline, dashed, color={rgb:red,2;black,1}] (stm1) edge[bend left] (st);
	\draw[myline, dashed, color={rgb:red,2;black,1}] (ot) edge[bend left] (st);
	\end{scope}
\end{tikzpicture}}
    \end{figure}
}
\only<3->{
    Части модели мира:
    
    \vspace{0.5cm}
    \footnotesize{\textbf{Генеративная} модель прямой причинности:}
    \scriptsize{
    \begin{itemize}
    \setlength\itemsep{1ex}
    \setlength\topsep{1.5ex}
        \item<3-> $s_t \sim p(s_t \mid s_{t-1}, a_{t-1})$ модель динамики скрытых состояний
        \item<4-> $u_t \sim p(u_t \mid s_t, c)$ модель прямой причинности
        \item<5-> $s'_t \sim p(s'_t\mid s'_{t-1}, u_t)$ модель обновления 
    \end{itemize} }
    \vspace{0.4cm}
    \footnotesize{Обратная причинная модель (\textbf{модель вывода})}:
    \scriptsize{
    \begin{itemize}
    \setlength\itemsep{1ex}
    \setlength\topsep{1.5ex}
        \item<3-> $s_t \sim q(s_t \mid s_{t-1}, a_{t-1}, o_t)$ модель вывода состояния робота 
        \item<8-> $s'_t \sim q(s'_t \mid o'_t)$ модель вывода объектного состояния
        \item<9-> $u_t \sim q(u_t \mid s_t, s'_t)$ модель обратной причинности
    \end{itemize} }
    \vspace{0.4cm}
    \footnotesize{Выходы нейронной сети}:
    \scriptsize{
    \begin{itemize}
    \setlength\itemsep{1ex}
    \setlength\topsep{1.5ex}
        \item<3-> $p(o_t \mid s_t)$ субъектный декодер 
        \item<5-> $p(o'_t \mid s'_t)$ объектный декодер
        \item<6-> $p(r_t \mid u_t)$ декодер награды
        \item<10->$\pi(a_t \mid s_t, c)$, $v(s_{t}, u_{t}, s'_{t})$ стратегия и функция полезности
    \end{itemize} }
    \vspace{0.1cm}
}
\end{column}

\begin{column}{0.48\linewidth}
\only<2-10>{
    Модель мира CEMA:
    \begin{figure}
        \resizebox{.9\linewidth}{!}{\begin{frame}
\frametitle{Модель мира CEMA}
% TODO: titles are hanging out of images
% TODO: maybe draw commentaries
% TODO: draw an arrow

\begin{columns}[b]
\begin{column}{0.48\linewidth}
\only<1-2>{
    Модель мира Dreamer\cite{Dreamer}:
    \vspace{0.45cm}
    
    \footnotesize{
    \begin{itemize}
    \setlength\itemsep{1ex}
    \setlength\topsep{1.5ex}
        \item $s_t \sim p(s_t \mid s_{t-1}, a_{t-1})$ модель динамики среды
        \item $s_t \sim q(s_t \mid s_{t-1}, a_{t-1}, o_t)$ модель вывода
        \item $p(o_t \mid s_t)$ модель изображения
        \item $p(r_t \mid s_t)$ модель награды
    \end{itemize} }
    \vspace{0.4cm}
    
    \begin{figure}
        \resizebox{.9\linewidth}{!}{\input{images/schemes/dreamer_wm}}
    \end{figure}
}
\only<3->{
    Части модели мира:
    
    \vspace{0.5cm}
    \footnotesize{\textbf{Генеративная} модель прямой причинности:}
    \scriptsize{
    \begin{itemize}
    \setlength\itemsep{1ex}
    \setlength\topsep{1.5ex}
        \item<3-> $s_t \sim p(s_t \mid s_{t-1}, a_{t-1})$ модель динамики скрытых состояний
        \item<4-> $u_t \sim p(u_t \mid s_t, c)$ модель прямой причинности
        \item<5-> $s'_t \sim p(s'_t\mid s'_{t-1}, u_t)$ модель обновления 
    \end{itemize} }
    \vspace{0.4cm}
    \footnotesize{Обратная причинная модель (\textbf{модель вывода})}:
    \scriptsize{
    \begin{itemize}
    \setlength\itemsep{1ex}
    \setlength\topsep{1.5ex}
        \item<3-> $s_t \sim q(s_t \mid s_{t-1}, a_{t-1}, o_t)$ модель вывода состояния робота 
        \item<8-> $s'_t \sim q(s'_t \mid o'_t)$ модель вывода объектного состояния
        \item<9-> $u_t \sim q(u_t \mid s_t, s'_t)$ модель обратной причинности
    \end{itemize} }
    \vspace{0.4cm}
    \footnotesize{Выходы нейронной сети}:
    \scriptsize{
    \begin{itemize}
    \setlength\itemsep{1ex}
    \setlength\topsep{1.5ex}
        \item<3-> $p(o_t \mid s_t)$ субъектный декодер 
        \item<5-> $p(o'_t \mid s'_t)$ объектный декодер
        \item<6-> $p(r_t \mid u_t)$ декодер награды
        \item<10->$\pi(a_t \mid s_t, c)$, $v(s_{t}, u_{t}, s'_{t})$ стратегия и функция полезности
    \end{itemize} }
    \vspace{0.1cm}
}
\end{column}

\begin{column}{0.48\linewidth}
\only<2-10>{
    Модель мира CEMA:
    \begin{figure}
        \resizebox{.9\linewidth}{!}{\input{images/schemes/cema_wm}}
    \end{figure}
}
\end{column}
\end{columns}

\note[item]<1->{Let's take a look at RSSM World Model used in Dreamer. It has a single dynamics model and a single latent state, representing the state of the whole environment.}
\note[item]<2->{In CEMA, we split dynamics model on two parts: robot dynamics and object dynamics.}
\note[item]<3->{For robot dynamics, we left the RSSM without any changes, since the robot state is controlled directly by actions.}
\note[item]<4->{From imagined robot latent state, we predict the task-specific influence vector, which represents the collateral object control performed by robot.}
\note[item]<5->{The next object state is predicted using RSSM and the imagined influence vector.}
\note[item]<6->{The influence vector is also used to predict step reward.}
\note[item]<7->{The inference model of actor remains unchanged to Dreamer, specifically, the robot state is inferred from its previous state and current robot observation.}
\note[item]<8->{Object latent state, on contrary, is inferred from the object image alone.}
\note[item]<9->{For influence vector, posterior distribution is conditioned on both object and actor latent states to reflect the changes made to object by actor.}
\note[item]<10->{Value function is conditioned on all latent states, while policy is conditioned only on robot state and low-dimensional task context vector.}
\end{frame}}
    \end{figure}
}
\end{column}
\end{columns}

\note[item]<1->{Let's take a look at RSSM World Model used in Dreamer. It has a single dynamics model and a single latent state, representing the state of the whole environment.}
\note[item]<2->{In CEMA, we split dynamics model on two parts: robot dynamics and object dynamics.}
\note[item]<3->{For robot dynamics, we left the RSSM without any changes, since the robot state is controlled directly by actions.}
\note[item]<4->{From imagined robot latent state, we predict the task-specific influence vector, which represents the collateral object control performed by robot.}
\note[item]<5->{The next object state is predicted using RSSM and the imagined influence vector.}
\note[item]<6->{The influence vector is also used to predict step reward.}
\note[item]<7->{The inference model of actor remains unchanged to Dreamer, specifically, the robot state is inferred from its previous state and current robot observation.}
\note[item]<8->{Object latent state, on contrary, is inferred from the object image alone.}
\note[item]<9->{For influence vector, posterior distribution is conditioned on both object and actor latent states to reflect the changes made to object by actor.}
\note[item]<10->{Value function is conditioned on all latent states, while policy is conditioned only on robot state and low-dimensional task context vector.}
\end{frame}}
    \end{figure}
}
\end{column}
\end{columns}

\note[item]<1->{Let's take a look at RSSM World Model used in Dreamer. It has a single dynamics model and a single latent state, representing the state of the whole environment.}
\note[item]<2->{In CEMA, we split dynamics model on two parts: robot dynamics and object dynamics.}
\note[item]<3->{For robot dynamics, we left the RSSM without any changes, since the robot state is controlled directly by actions.}
\note[item]<4->{From imagined robot latent state, we predict the task-specific influence vector, which represents the collateral object control performed by robot.}
\note[item]<5->{The next object state is predicted using RSSM and the imagined influence vector.}
\note[item]<6->{The influence vector is also used to predict step reward.}
\note[item]<7->{The inference model of actor remains unchanged to Dreamer, specifically, the robot state is inferred from its previous state and current robot observation.}
\note[item]<8->{Object latent state, on contrary, is inferred from the object image alone.}
\note[item]<9->{For influence vector, posterior distribution is conditioned on both object and actor latent states to reflect the changes made to object by actor.}
\note[item]<10->{Value function is conditioned on all latent states, while policy is conditioned only on robot state and low-dimensional task context vector.}
\end{frame}}
    \end{figure}
}
\end{column}
\end{columns}

\note[item]<1->{Let's take a look at RSSM World Model used in Dreamer. It has a single dynamics model and a single latent state, representing the state of the whole environment.}
\note[item]<2->{In CEMA, we split dynamics model on two parts: robot dynamics and object dynamics.}
\note[item]<3->{For robot dynamics, we left the RSSM without any changes, since the robot state is controlled directly by actions.}
\note[item]<4->{From imagined robot latent state, we predict the task-specific influence vector, which represents the collateral object control performed by robot.}
\note[item]<5->{The next object state is predicted using RSSM and the imagined influence vector.}
\note[item]<6->{The influence vector is also used to predict step reward.}
\note[item]<7->{The inference model of actor remains unchanged to Dreamer, specifically, the robot state is inferred from its previous state and current robot observation.}
\note[item]<8->{Object latent state, on contrary, is inferred from the object image alone.}
\note[item]<9->{For influence vector, posterior distribution is conditioned on both object and actor latent states to reflect the changes made to object by actor.}
\note[item]<10->{Value function is conditioned on all latent states, while policy is conditioned only on robot state and low-dimensional task context vector.}
\end{frame}